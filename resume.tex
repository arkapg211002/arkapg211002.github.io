\documentclass{resume} % Use the custom resume.cls style

\usepackage[left=0.4 in,top=0.3in,right=0.4 in,bottom=0.3in]{geometry} % Document margins
\newcommand{\tab}[1]{\hspace{.2667\textwidth}\rlap{#1}} 
\newcommand{\itab}[1]{\hspace{0em}\rlap{#1}}
\name{ARKAPRATIM GHOSH} % Your name
% You can merge both of these into a single line, if you do not have a website.
\address{+91 9330450430 \\ Kolkata, India} 
\address{\href{mailto:contact@faangpath.com}{arkapratimghosh1264@gmail.com} \\ \href{https://www.linkedin.com/in/arkapratim-ghosh21102002/}{LinkedIn: arkapratim-ghosh21102002} \\ 
\href{https://arkapg211002.github.io/}{Portfolio} \\
\href{https://github.com/arkapg211002}{Github: arkapg211002}}  %

\begin{document}

%----------------------------------------------------------------------------------------
%	OBJECTIVE
%----------------------------------------------------------------------------------------

\begin{rSection}{OBJECTIVE}

{Seeking an IT position to leverage my academic foundation in B.Tech CSE (expected 2025) and develop practical skills, contributing to organizational success in software development and technology innovation.}


\end{rSection}
%----------------------------------------------------------------------------------------
%	EDUCATION SECTION
%----------------------------------------------------------------------------------------

\begin{rSection}{Education}

{\bf Bachelor of Technology in Computer Science and Engineering}, Techno Main Salt Lake \hfill {2021 - 2025}\\
Relevant Coursework: Data Structures and Algorithms, Computer Organization and Architecture, Operating Systems, Software Engineering, Compiler Design, Computer Networks, Database Management System

{\bf Council for the Indian School Certificate Examinations}, Pramila Memorial Institute\hfill {2005 - 2021}
%Minor in Linguistics \smallskip \\
%Member of Eta Kappa Nu \\
%Member of Upsilon Pi Epsilon \\


\end{rSection}

%----------------------------------------------------------------------------------------
% TECHINICAL STRENGTHS	
%----------------------------------------------------------------------------------------
\begin{rSection}{SKILLS}

\begin{tabular}{ @{} >{\bfseries}l @{\hspace{6ex}} l }
Technical Skills & Core Java, C, Python, Shell Scripting, HTML, CSS, JavaScript, Bootstrap, MERN Stack, 
\\ & PHP, MySQL, Unified Modelling Language.
\\
Soft Skills & Communication, Problem Solving, Time Management, Creativity, Adaptability

\end{tabular}\\
\end{rSection}

\begin{rSection}{EXPERIENCE}

\textbf{Open-source Contributor} \hfill October 2022\\
Hacktoberfest : Made pull requests and got all those accepted\hfill \textit{India}
 
 
\textbf{Poster Designer} \hfill March 2022 - July 2022\\
Samarth, Techno Main Salt Lake : Made posters that got featured on some occasions\hfill \textit{Kolkata, India}
 

\end{rSection} 

%----------------------------------------------------------------------------------------
%	WORK EXPERIENCE SECTION
%----------------------------------------------------------------------------------------

\begin{rSection}{PROJECTS}
\vspace{-1.1em}
\item \textbf{PlayNet} {PlayNet is a YouTube clone web application developed using HTML, CSS, JavaScript, JSX, ReactJS, and React Router DOM. It integrates with the YouTube API v3 from Rapid API to provide users with a similar experience to YouTube. PlayNet allows users to browse and watch videos, explore different genres, and search for specific channels. It fetches the latest 20 videos from the API on the home page and utilizes rate limits to ensure efficient API usage. Discover and enjoy a wide range of content on PlayNet, all within a user-friendly interface \href{https://playnet-sandy.vercel.app/}{(Try it here)}}
\item \textbf{Portfolio Website} {My personal portfolio is made using HTML, CSS, JavaScript, Bootstrap, Typed.js, LocalStorage API, Count API, vanilla-tilt.js, and AOS animation. HTML, CSS, and Bootstrap are used for designing the components. JavaScript is used for DOM, opening my terminal webpage in another browser, and showing typing text using Typed.js. AOS is used for adding animations when components are visible by scrolling. LocalStorage API is used to store the current theme on the visitor’s browser and apply it whenever he/she visits the website again. Count API is used to show the number of views on the website. Vanilla-tilt.js is used for adding tilting components when hovered over
\href{https://arkapg211002.github.io/}{(Try it here)}}
\item \textbf{Student Result Management System} {The Student Result Management System is a comprehensive and efficient software solution designed to streamline the process of managing and organizing student academic results in an educational institution. The project is made using HTML, CSS, JavaScript, Bootstrap, jQuery, PHP, MySQL \href{https://13000121058arkapratimghoshsrms.000webhostapp.com/}{(Try it here)}}
\end{rSection} 

%----------------------------------------------------------------------------------------
\begin{rSection}{ACHIEVEMENTS} 
\begin{itemize}
    \item 	AZ-900 \href{https://learn.microsoft.com/api/credentials/share/en-in/arkapratimghosh-4932/AB85DDF780F0F7DA?sharingId=EEAD82A39A0C4FAC}{certified}
    \item	Secured rank 2993 in WBJEE and AIR 40316 in JEE Mai
\end{itemize}


\end{rSection}

%----------------------------------------------------------------------------------------



\end{document}
